\section{MOTIVAATIO}
    Talvella 2023 käytyjen, peliaikoja ruotivien Telegram-myllyjen lomassa näin yksittäisiä viestejä,
    jotka pohtivat pelien kestoon vaikuttavia syitä.
    Vaikka toisten pelaajien syyttäminen väärästä pelityylistä ja etenkin yleisestä pahuudesta onkin mukavaa
    \textendash{} \textbackslash{}s, kappa ja mitä näitä nyt on \textendash{},
    saadaan mahdolliset aikatauluongelmat ratkaistua paremmin, kun niiden syyt ymmärretään.
    Asia taisi talvella jäädä vain keskusteluksi,
    mutta tähän eivät mielestäni riitä mutuilu ja mielipiteet vaan tarvitaan kovia keinoja.

    Aineiston kerääminen toki on aikaa vievää ja työlästä, mutta tarpeeksi tylsänä ihmisenä tähän ryhdyin.
    Ihan kivaa se loppujen lopuksi oli.
    Aineistoa analysoidessa sain myös tilaisuuden opetella uusien Python-pakettien käyttöä
    ja \LaTeX{}in kirjoittamista muuallakin kuin Overleafissa,
    joten ehkä tästä oli jotain oikeaa iloakin.

    Ehkä tätä analyysia voi käyttää turnauksien ja sarjojen aikataulujen suunnitteluun.
    Ainakin tästä saa tietoa siitä, millaista vaihtelua pelien kestoissa on järjestäjästä riippumattomista syistä.
    Ja ehkä halukkaat voivat saada tästä keinoja, joilla nopeuttaa omaa pelitahtiaan.
    Kenenkään syyllistäminen hitaudesta ei kuitenkaan ollut tarkoituksena
    enkä toivo että kukaan tätä sellaisena kokee.
    Toivottavasti tästä on edes hetkellistä iloa jollekin.