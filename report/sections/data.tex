\section{Aineisto}
    Tämän raportion aineisto on kerätty Futuliigan livestriimeistä vuodelta 2023.
    Tallenteista on merkattu ylös niillä näkyvien heittojen aikaleimat ja ajat, jolloin konat on saatu pystyyn.
    Jokaisesta heitosta on merkattu myös heiton heittäjä.

    Sekä heittojen että konankasauksien aikaleimojen tarkkuus on sekunti.
    Johtuen mm. videoiden joskus puutteellisesta tarkkuudesta, kamerakulmista johtuvista rajotteista ja YouTubessa satunnaisesti esiintyvästä yksittäisten kuvien toistosta, aikaleimat eivät ole täysin tarkkoja.
    Heitoissa on pyritty merkitsemään aikaleimaksi se kuva, jolloin karttu on viimeisen kerran heittäjän kädessä.
    Kona katsottiin kasatuksi, kun edellisessä erässä pelannut joukkue asettaa viimeisen kyykän paikalleen tai viimeisen tornin kohdalleen.
    Esimerkiksi peliä aloittamaan tulevan joukkueen tekemiä korjauksia ei siis laskettu mukaan konankasausaikaan.

    Yksittäisten heittojen aikaleimoja kerättiin \textcolor{red}{N} kappaletta ja konankasauksia \textcolor{red}{N} kappaletta.
    Analyysin kohteena oli heittäjän heittoon ja joukkueen konankasaamiseen käyttämän aika, joten raakojen aikaleimojen sijaan analyysissa käytettiin niiden välisiä aikoja.
    Analyysissa huomioitiin kaikki konankasaukset, mutta \textcolor{red}{N} heittoväliä jätettiin analyysista pois,
    koska suurena tekijänä heittojen väliseen aikaan on pelien keskinäinen synkronisointi eikä sen mallintamista koettu mielekkääksi.